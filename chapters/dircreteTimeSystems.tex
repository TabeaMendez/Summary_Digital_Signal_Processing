% 
% (c) Copyright 2016 Tabea Mendez
% 
% This source is free: you can redistribute it and/or modify
% it under the terms of the GNU General Public License as published by
% the Free Software Foundation, either version 3 of the License, or
% (at your option) any later version.
% 
% This source is distributed in the hope that it will be useful,
% but WITHOUT ANY WARRANTY; without even the implied warranty of
% MERCHANTABILITY or FITNESS FOR A PARTICULAR PURPOSE.  See the
% GNU General Public License for more details.
% 
% You should have received a copy of the GNU General Public License
% along with this source.  If not, see <http://www.gnu.org/licenses/>.
%
%%%%%%%%%%%%%%%%%%%%%%%%%%%%%%%%%%%%%%%%%%%%%%%%%%%%%%%%%%%%%%%%%%%%%%%%%%%%%%

	\begin{minipage}{0.6\textwidth}
		Ein Discrete-Time-System ist ein System, welches eine Eingangssequenz in eine Ausgangssequenz transformiert.\\
		Die Transformation kann dabei auf zwei Arten erfolgen:
		\begin{itemize}
		 \item Sample für Sample
		 \item Blockweise
		\end{itemize}
	\end{minipage}
	\begin{minipage}{0.4\textwidth}
		\begin{tikzpicture}[>=latex', scale=1.5]
			\draw[line width=1](-0.75,-0.5)--(0.75,-0.5)--(0.75,0.5)--node[above]{\footnotesize Discrete-Time-System}(-0.75,0.5)--cycle;
			\draw[line width=1,->](-1.8,0)--node[above]{\footnotesize$x(n)$}node[below]{\footnotesize\parbox{2cm}{\centering input Sequenz}}(-0.75,0);
			\draw[line width=1,<-](1.8,0)--node[above]{\footnotesize$y(n)$}node[below]{\footnotesize\parbox{2cm}{\centering output Sequenz}}(0.75,0);
			\node at (0,0){\huge$H$};
		\end{tikzpicture} 
	\end{minipage}\\

	\begin{minipage}{0.45\textwidth}
		\textbf{Sample für Sample}\\[0.2cm]
		$\begin{array}{c}
		\{x_0,x_1,x_2,...,x_n,...\}\\ \downarrow H\\
		\{y_0,y_1,y_2,...,y_n,...\}
		\end{array}$\\[0.2cm]
		\textbf{Bsp:}\\[0.1cm]
		$y(n) = 2\,x(n) + 3\,x(n-1) + 4\, x(n-2)$
	\end{minipage}
	\begin{minipage}{0.25\textwidth}
		\textbf{Blockweise}\\[0.2cm]
		$\begin{bmatrix}x_0\\x_1\\\vdots\\x_n\\\vdots\end{bmatrix}\xrightarrow H \begin{bmatrix}y_0\\y_1\\\vdots\\y_n\\\vdots\end{bmatrix}$\\[0.2cm]
	\end{minipage}
	\begin{minipage}{0.25\textwidth}
		\textbf{Bsp:}$\quad\vec y = H\cdot \vec x$\\[0.2cm]
		$\begin{bmatrix}y_0\\y_1\\y_2\\y_3\\y_4\end{bmatrix} =\begin{bmatrix}2 & 0 & 0\\3 & 2 & 0\\4 & 3 & 2\\0 & 4 & 3 \\0 & 0 & 4\end{bmatrix}\cdot \begin{bmatrix}x_0\\x_1\\x_2\end{bmatrix}$
	\end{minipage}
	
\section{Linearität und Zeitinvarianz}
	\textbf{Linearität:}\\[0.2cm]
		Für ein lineares System muss die Superpositionsbedingung erfüllt sein:\\[0.2cm]
		\fcolorbox{CadetRed}{white}{$y(n) = a_1\,H[x_1(n)]+ a_2\,H[x_2(n)]= H[a_1\,x_1(n) + a_2\,x_2(n)] $}\\[0.2cm]
	\begin{tikzpicture}[>=latex', scale=1.4]
		\def\s{0.3};
		\def\l{1};
		\def\r{0.15};
		\def\dis{0.8};

		\coordinate (h1) at (0,\dis);
		\draw[line width=1](h1)++(-\s,-\s)--++(2*\s,0)--++(0,2*\s)--++(-2*\s,0)--cycle node at(h1){$H$};
		\draw[line width=1,->](h1)++(-\l-\s,0)--node[above]{\footnotesize$x_1(n)$}++(\l,0);
		\draw[line width=1,->](h1)++(\s,0)--node[above]{\footnotesize$y_1(n)$}++(\l,0);
		\draw[line width=1](h1)++(\s+\l,0)node[right]{$a_1$}--++(0,\s)--++(2*\s,-\s)--++(-2*\s,-\s)--cycle;
		\draw[line width=1,->](h1)++(3*\s+\l,0)--++(\l/2,0)--++(0,-\dis+\r);

		\draw[line width=1,->](h1)++(3*\s+3/2*\l,-\dis)circle(\r)node{$+$};
		\draw[line width=1,->](h1)++(3*\s+3/2*\l+\r,-\dis)--node[above]{$y(n)$}++(\l,0);

		\coordinate (h2) at (0,-\dis);
		\draw[line width=1](h2)++(-\s,-\s)--++(2*\s,0)--++(0,2*\s)--++(-2*\s,0)--cycle node at(h2){$H$};
		\draw[line width=1,->](h2)++(-\l-\s,0)--node[above]{\footnotesize$x_2(n)$}++(\l,0);
		\draw[line width=1,->](h2)++(\s,0)--node[above]{\footnotesize$y_2(n)$}++(\l,0);
		\draw[line width=1](h2)++(\s+\l,0)node[right]{$a_2$}--++(0,\s)--++(2*\s,-\s)--++(-2*\s,-\s)--cycle;
		\draw[line width=1,->](h2)++(3*\s+\l,0)--++(\l/2,0)--++(0,\dis-\r);
	\end{tikzpicture}$\qquad\qquad$
	\begin{tikzpicture}[>=latex', scale=1.4]
		\def\s{0.3};
		\def\l{1};
		\def\r{0.15};
		\def\dis{0.8};

		\coordinate (h1) at (0,\dis);
		\draw[line width=1,->](h1)++(-\l-\s,0)--node[above]{\footnotesize$x_1(n)$}++(\l,0);
		\draw[line width=1](h1)++(-\s,0)node[right]{$a_1$}--++(0,\s)--++(2*\s,-\s)--++(-2*\s,-\s)--cycle;
		\draw[line width=1,->](h1)++(\s,0)--++(\l/2,0)--++(0,-\dis+\r);

		\draw[line width=1,->](h1)++(\s+1/2*\l,-\dis)circle(\r)node{$+$};
		\draw[line width=1,->](h1)++(\s+1/2*\l+\r,-\dis)--node[above]{$x(n)$}++(\l,0);

		\coordinate (h3) at (3/2*\l+\r+2*\s,0);
		\draw[line width=1](h3)++(-\s,-\s)--++(2*\s,0)--++(0,2*\s)--++(-2*\s,0)--cycle node at(h3){$H$};
		\draw[line width=1,->](h3)++(\s,0)--node[above]{\footnotesize$y(n)$}++(\l,0);

		\coordinate (h2) at (0,-\dis);
		\draw[line width=1,->](h2)++(-\l-\s,0)--node[above]{\footnotesize$x_2(n)$}++(\l,0);
		\draw[line width=1](h2)++(-\s,0)node[right]{$a_2$}--++(0,\s)--++(2*\s,-\s)--++(-2*\s,-\s)--cycle;
		\draw[line width=1,->](h2)++(\s,0)--++(\l/2,0)--++(0,\dis-\r);
	\end{tikzpicture}
	
	\textbf{Zeitinvarianz:}\\[0.2cm]
	Ein Zeitinvariantes System ändert seine Eigenschaften über die Zeit nicht.$\qquad$
	\fcolorbox{CadetRed}{white}{$y_D(n) = y(n-D)$}\\[0.2cm]
	\begin{tikzpicture}[>=latex', scale=1.4]
		\def\s{0.3};
		\def\l{1};
		\def\d{1};

		\coordinate (h3) at (0,0);
		\draw[line width=1,->](h3)++(-\l-\s,0)--node[above]{\footnotesize$x(n)$}++(\l,0);
		\draw[line width=1](h3)++(-\s,-\s)--++(2*\s,0)--++(0,2*\s)--++(-2*\s,0)--cycle node at(h3){$H$};
		\draw[line width=1,->](h3)++(\s,0)--node[above]{\footnotesize$y(n)$}++(\l,0);
		\draw[line width=1](h3)++(\s+\l,-\s)--++(2*\s,0)--++(0,2*\s)--++(-2*\s,0)--cycle node at(2*\s+\l,0){$D$};
		\draw[line width=1,->](h3)++(3*\s+\l,0)--node[above]{\footnotesize$y(n-D)$}++(\l,0);

		\coordinate (h4) at (0,-\d);
		\draw[line width=1,->](h4)++(-\l-\s,0)--node[above]{\footnotesize$x(n)$}++(\l,0);
		\draw[line width=1](h4)++(-\s,-\s)--++(2*\s,0)--++(0,2*\s)--++(-2*\s,0)--cycle node at(h4){$D$};
		\draw[line width=1,->](h4)++(\s,0)--node[above]{\footnotesize$x(n-D)$}node[below]{\footnotesize$x_D(n)$}++(\l,0);
		\draw[line width=1](h4)++(\s+\l,-\s)--++(2*\s,0)--++(0,2*\s)--++(-2*\s,0)--cycle node at(2*\s+\l,-\d){$H$};
		\draw[line width=1,->](h4)++(3*\s+\l,0)--node[above]{\footnotesize$y_D(n)$}++(\l,0);
	\end{tikzpicture}$\qquad\qquad$ 
	\begin{tikzpicture}[>=latex', scale=1.4]
		\def\s{0.4};
		\def\l{1};
		\def\d{0.4};

		\coordinate (h3) at (0,0);
		\draw[line width=1,->](h3)++(-\l-\s,0)--node[above]{$x(n)$}++(\l,0);

		\draw[line width=1](h3)++(-\s,-\s)--++(2*\s,0)--++(0,2*\s)--++(-2*\s,0)--cycle node at(h3){\parbox{2cm}{\centering  {\footnotesize Delay}\\ $D$}};
		\draw[line width=1,->](h3)++(\s,0)--node[above, xshift=6pt]{$x(n-D)$}++(\l,0);

		\coordinate (f1) at (-3,-\s);
		\draw[line width=0.5,->](f1)++(-0.1,0)--++(1.5*\l+0.2,0)node[right]{\footnotesize$n$};
		\draw[line width=0.5,->](f1)++(0,-0.1)node[below]{\tiny$0$}--++(0,\l)node[right]{\footnotesize$x(n)$};
		\foreach \i in {1,...,6}
		\draw[line width=0.5,fill](f1)++({(\i-1)/4},0)--++(0,{0.3*cos(\i*0.55*180/pi)+0.4})circle(0.03);

		\coordinate (f2) at (2.1,-\s);
		\draw[line width=0.5,->](f2)++(-0.1,0)--++(1.5*\l+\d+0.2,0)node[right]{\footnotesize$n$};
		\draw[line width=0.5,->](f2)++(0,-0.1)node[below]{\tiny$0$}--++(0,\l)node[right]{\footnotesize$x(n-D)$};
		\draw[line width=0.5,<->](f2)++(0,0.3)--node[above]{\tiny$D$}++(\d,0);

		\foreach \i in {1,...,6}
		\draw[line width=0.5,fill](f2)++({(\i-1)/4+\d},0)--++(0,{0.3*cos(\i*0.55*180/pi)+0.4})circle(0.03);
	\end{tikzpicture}\\[-0.3cm]

\section{Impulsantwort}
\vspace*{-0.6cm}\begin{minipage}{0.35\textwidth}
	Ein lineares, zeitinvariantes System, ist durch seine Impulsantwort vollständig bestimmt. Wird am Eingang ein Impuls $\delta(n)$ angelet, so antwortet das System mit der Impulsantwort $h(n)$.
\end{minipage}\begin{minipage}{0.05\textwidth}$ $\end{minipage}
	\begin{minipage}{0.65\textwidth}
			\begin{tikzpicture}[>=latex', scale=1.4]
			\def\s{0.4};
			\def\l{1};
			\def\d{0};

			\coordinate (h3) at (0,0);
			\draw[line width=1,->](h3)++(-\l-\s,0)--node[above]{$\delta(n)$}++(\l,0);

			\draw[line width=1](h3)++(-\s,-\s)--++(2*\s,0)--++(0,2*\s)--++(-2*\s,0)--cycle node at(h3){$H$};
			\draw[line width=1,->](h3)++(\s,0)--node[above]{$h(n)$}++(\l,0);

			\coordinate (f1) at (-2.3,-\s);
			\draw[line width=0.5,->](f1)++(-0.5,0)--++(1,0)node[right]{\footnotesize$n$};
			\draw[line width=0.5,->](f1)++(0,-0.1)node[below]{\tiny$0$}--++(0,\l)node[right]{\footnotesize$\delta(n)$};
			\foreach \i in {1}
			\draw[line width=1,fill](f1)++({(\i-1)/4},0)--++(0,{0.25*cos(\i*0.55*180/pi)+0.4})circle(0.04);

			\coordinate (f2) at (2.1,-\s);
			\draw[line width=0.5,->](f2)++(-0.5,0)--++(1.5*\l+\d+0.6,0)node[right]{\footnotesize$n$};
			\draw[line width=0.5,->](f2)++(0,-0.1)node[below]{\tiny$0$}--++(0,\l)node[right]{\footnotesize$h(n)$};
	% 		\draw[line width=0.5,<->](f2)++(0,0.3)--node[above]{\tiny$D$}++(\d,0);

			\foreach \i in {1,...,6}
			{
				\draw[line width=1,fill](f2)++({(\i-1)/4+\d},0)--++(0,{0.3*cos(\i*0.55*180/pi)+0.4})circle(0.04);
			}
		\end{tikzpicture} 

		\begin{tikzpicture}[>=latex', scale=1.4]
			\def\s{0.4};
			\def\l{1};
			\def\d{0.6};

			\coordinate (h3) at (0,0);
			\draw[line width=1,->](h3)++(-\l-\s,0)--node[above,, xshift=-5pt]{$\delta(n-k)$}++(\l,0);

			\draw[line width=1](h3)++(-\s,-\s)--++(2*\s,0)--++(0,2*\s)--++(-2*\s,0)--cycle node at(h3){$H$};
			\draw[line width=1,->](h3)++(\s,0)--node[above, xshift=5pt]{$h(n-k)$}++(\l,0);

			\coordinate (f1) at (-2.3,-\s);
			\draw[line width=0.5,->](f1)++(-0.5,0)--++(1,0)node[right]{\footnotesize$n$};
			\draw[line width=0.5,->](f1)++(-0.4,-0.1)node[below]{\tiny$0$}--++(0,\l)node[right]{\footnotesize$\delta(n-k)$};
			\foreach \i in {1}
			{
			\draw[line width=1,fill](f1)++({(\i-1)/4+0.2},0)node[below]{$k$}--++(0,{0.25*cos(\i*0.55*180/pi)+0.4})circle(0.04);
			}
			\coordinate (f2) at (2.1,-\s);
			\draw[line width=0.5,->](f2)++(-0.5,0)--++(1.5*\l+\d+0.6,0)node[right]{\footnotesize$n$};
			\draw[line width=0.5,->](f2)++(0,-0.1)node[below]{\tiny$0$}--++(0,\l)node[right]{\footnotesize$h(n-k)$};
	% 		\draw[line width=0.5,<->](f2)++(0,0.3)--node[above]{\tiny$D$}++(\d,0);

			\foreach \i in {1}
			{
				\draw[line width=1,fill](f2)++({(\i-1)/4+\d},0)node[below]{$k$}--++(0,{0.3*cos(\i*0.55*180/pi)+0.4})circle(0.04);
			}
			\foreach \i in {2,...,6}
			{
				\draw[line width=1,fill](f2)++({(\i-1)/4+\d},0)node[below]{$ $}--++(0,{0.3*cos(\i*0.55*180/pi)+0.4})circle(0.04);
			}
		\end{tikzpicture}
	\end{minipage}\\[-0.2cm]
	\begin{description}
		\item [LTI-Form:] $ $\\
		\begin{minipage}[t]{0.65\textwidth}
		Eingangssequenz ist Folge von gewichteten und verzögerten Diracs.\\
		Ausgangssequenz ist Überlagerung der gewichteten und verzögerten Impulsantworten.
		\end{minipage}\begin{minipage}[t]{0.03\textwidth}$ $\end{minipage}
		\begin{minipage}[t]{0.28\textwidth}
		$ $\\[-0.35cm]\fcolorbox{CadetRed}{white}{$y(n) = \mysum{m}{}{x(m)\cdot h(n-m)}$}\\
		\end{minipage}
		\item [Direkt Form:]$ $\\
		\begin{minipage}{\textwidth}
		Faltung der Eingangssequenz mit der Impulsantwort$\qquad\qquad\qquad\qquad\quad\;\,$
		\fcolorbox{CadetRed}{white}{$y(n) = \mysum{m}{}{h(m)\cdot x(n-m)}$}
		\end{minipage}
	\end{description}

\section{FIR und IIR Filter}
	Diskrete LTI-Systeme werden über ihre Impulsantwort klassifiziert\\[0.2cm]
	\begin{minipage}{0.45\textwidth}
		\begin{itemize}
		\item \textbf{FIR Filter:} Finite Impulse Response\\[-0.2cm]
		\item \textbf{IIR Filter:} Infinite Impulse Response\\[0.4cm]
		\end{itemize}
	\end{minipage}
	\begin{minipage}{0.5\textwidth}
		\begin{tikzpicture}[>=latex', scale=1.6]
			\def\s{0.4};
			\def\l{1};
			\def\d{0};

			\coordinate (f2) at (0,0);
			\draw[line width=0.5,->](f2)++(-0.4,0)--++(2.1*\l+\d+0.6,0)node[right]{\footnotesize$n$};
			\draw[line width=0.5,->](f2)++(0,-0.1)node[below]{\footnotesize$0$}--++(0,\l)node[right]{\footnotesize FIR $h(n)$};
	% 		\draw[line width=0.5,<->](f2)++(0,0.3)--node[above]{\tiny$D$}++(\d,0);

			\foreach \i in {1,...,5}
			{
				\draw[line width=1,fill](f2)++({(\i-1)/4+\d},0)--++(0,{0.3*cos(\i*0.55*180/pi)+0.4})circle(0.04);
			}
			\foreach \i in {6}
			{
				\draw[line width=1,fill](f2)++({(\i-1)/4+\d},0)node[below]{\footnotesize$M$}--++(0,{0.3*cos(\i*0.55*180/pi)+0.4})circle(0.04);
			}
			\foreach \i in {0,7,8,9}
			{
				\draw[line width=1,fill](f2)++({(\i-1)/4+\d},0)--++(0,{0})circle(0.04);
			}

			\coordinate (f2) at (3.5,0);
			\draw[line width=0.5,->](f2)++(-0.4,0)--++(2.1*\l+\d+0.6,0)node[right]{\footnotesize$n$};
			\draw[line width=0.5,->](f2)++(0,-0.1)node[below]{\footnotesize$0$}--++(0,\l)node[right]{\footnotesize IIR $h(n)$};
	% 		\draw[line width=0.5,<->](f2)++(0,0.3)--node[above]{\tiny$D$}++(\d,0);

			\foreach \i in {1,...,6}
			{
				\draw[line width=1,fill](f2)++({(\i-1)/4+\d},0)--++(0,{0.3*cos(\i*0.55*180/pi)+0.4})circle(0.04);
			}
			\foreach \i in {0}
			{
				\draw[line width=1,fill](f2)++({(\i-1)/4+\d},0)--++(0,{0})circle(0.04);
			}
			\foreach \i in {8,9,10}
			{
				\draw[line width=1,fill](f2)++({(\i-1)/4.65+\d},0)++(0,{0.1})circle(0.015);
			}
		\end{tikzpicture} 
	\end{minipage}\\[-0.6cm]
	
	\begin{description}
	 \item [FIR Filter:]$ $\\[0.1cm]
		Ein FIR Filter hat eine endliche Anzahl Koeffizienten, die nicht Null sind.\\[0.1cm]
		\begin{tabular}{ll}
		 Filterkoeffizienten & \fcolorbox{CadetRed}{white}{$\{h_0,h_1,h_2,...,h_M,0,0,0,...\}$}\\[0.2cm]
		 Ordung: & \fcolorbox{CadetRed}{white}{$M$}\\[0.1cm]
		 Filterlänge: & \fcolorbox{CadetRed}{white}{$L_h = M + 1$}\\
		\end{tabular}\\[0.2cm]
		Für FIR Filter vereinfacht sich die Direkte Faltungsform zu:\\[0.1cm]
		$y(n) = h_0\, x(n) + h_1\, x(n-1) + h_2\, x(n-2) + ...  + h_M\, x(n-M)$\\[0.2cm]
		\fcolorbox{CadetRed}{white}{$y(n) = \mysum{m=0}{M}{h(m)\cdot x(n-m)}$}\\[0.1cm]
	 \item [IIR Filter:]$ $\\[0.1cm]
		Ein IIR Filter hat eine unendliche Anzahl Koeffizienten, die nicht Null sind. Da solche Filter nicht realisiert werden können, wird er Fokus auf eine wichtige Unterklasse der IIR Filter gelegt:\\ \textbf{IIR Filter die den Ausgang zurückführen (Feedback-Loop) }\\[0.1cm]
		Die Impulsantwort solcher IIR Filter kann in folgender Form geschrieben werden:\\[0.2cm]
		\fcolorbox{CadetRed}{white}{$h(n) = \underbrace{\mysum{i=1}{M}{a_i\,h(n-i)}}_{\text{IIR-Teil}}+ \underbrace{\mysum{j=0}{L}{b_j\,\delta(n-j)}}_{\text{FIR-Teil}}$}\\[0.2cm]
		und die Ein-Ausgangsgleichung folgendermassen:\\[0.1cm]
		$y(n) = \underbrace{a_1\,y(n-1) + a_2\,y(n-2) + ... + a_M\,y(n-M)}_{\text{IIR-Teil}} + \underbrace{b_0\, x(n) + b_1\, x(n-1) + b_2\, x(n-2) + ...  + b_L\, x(n-L)}_{\text{FIR-Teil}}$\\[0.2cm]
		\fcolorbox{CadetRed}{white}{$y(n) = \underbrace{\mysum{i=1}{M}{a_i\,y(n-i)}}_{\text{IIR-Teil}}+ \underbrace{\mysum{j=0}{L}{b_j\,x(n-j)}}_{\text{FIR-Teil}}$}\\[0.2cm]
	\end{description}
		
\section{Kausalität und Stabilität}
	\textbf{Kausalität}\\[0.2cm]
	Es gibt drei Arten von Signalen und demzufolge auch drei Arten von LTI-Systemen.\\[-0.65cm]
	\begin{itemize}
	 \item Kausale Signale / LTI-Systeme\\[-0.65cm]
	 \item Antikausale Signale / LTI-Systeme\\[-0.65cm]
	 \item Gemischte Signale / LTI-Systeme
	\end{itemize}

	\begin{tabularx}{\textwidth}{|X|X|X|}
	 \hline&&\\[-0.4cm]
		\textbf{Kausal} & \textbf{Antikausal} & \textbf{Gemischt}\\
	 \hline
		\begin{tikzpicture}[>=latex', scale=1.4]
			\def\s{0.4};
			\def\l{1};
			\def\d{0};

			\coordinate (f2) at (0,0);
			\draw[line width=0.5,->](f2)++(-1.5,0)--++(3.1,0)node[right]{\footnotesize$n$};
			\draw[line width=0.5,->](f2)++(0,-0.1)node[below]{\footnotesize$0$}--++(0,\l+0.1)node[right]{\footnotesize $x(n)$};
			\foreach \i in {1,...,6}
			{
				\draw[line width=1,fill](f2)++({(\i-1)/4+\d},0)--++(0,{0.3*cos(\i*0.55*180/pi)+0.4})circle(0.04);
			}
			\foreach \i in {-4,...,0}
			{
				\draw[line width=1,fill](f2)++({(\i-1)/4+\d},0)--++(0,{0})circle(0.04);
			}
		\end{tikzpicture}&		
		\begin{tikzpicture}[>=latex', scale=1.4]
			\def\s{0.4};
			\def\l{1};
			\def\d{0};
			
			\coordinate (f2) at (4,0);
			\draw[line width=0.5,->](f2)++(-1.5,0)--++(3.1,0)node[right]{\footnotesize$n$};
			\draw[line width=0.5,->](f2)++(0,-0.1)node[below]{\footnotesize$0$}--++(0,\l+0.1)node[right]{\footnotesize $x(n)$};

			\foreach \i in {-4,...,0}
			{
				\draw[line width=1,fill](f2)++({(\i-1)/4+\d},0)--++(0,{0.3*cos(\i*0.55*180/pi)+0.4})circle(0.04);
			}
			\foreach \i in {1,...,6}
			{
				\draw[line width=1,fill](f2)++({(\i-1)/4+\d},0)--++(0,{0})circle(0.04);
			}

		\end{tikzpicture}&		
		\begin{tikzpicture}[>=latex', scale=1.4]
			\def\s{0.4};
			\def\l{1};
			\def\d{0};

			\coordinate (f2) at (8,0);
			\draw[line width=0.5,->](f2)++(-1.5,0)--++(3.1,0)node[right]{\footnotesize$n$};
			\draw[line width=0.5,->](f2)++(0,-0.1)node[below]{\footnotesize$0$}--++(0,\l+0.1)node[right]{\footnotesize $x(n)$};

			\foreach \i in {-4,...,6}
			{
				\draw[line width=1,fill](f2)++({(\i-1)/4+\d},0)--++(0,{0.3*cos(\i*0.55*180/pi)+0.4})circle(0.04);
			}
		\end{tikzpicture}\\
	 \hline&&\\[-0.35cm]
		ungleich Null für $n>=0$ & ungleich Null für $n<=-1$  & ungleich Null alle $n$\\[0.05cm]
	 \hline
	\end{tabularx}
	
	\textbf{Stabilität:}\\[0.2cm]
	Ein LTI-System ist stabil, wenn ein begrenzter Eingang nur einen begrenzten Ausgang generiert.\\[0.2cm]
	\fcolorbox{CadetRed}{white}{$\mysum{n=-\infty}{\infty}{|h(n)|}<\infty$}\\
	
	\textbf{Stabilität und Kausalität:}
	\begin{itemize}
	 \item Stabilität und Kausalität sind unabhänging voneinander
	 \item Stabilität ist zwingend Kausalität nicht
	 \item Um ein stabiles, antikausales oder gemischtes LTI-System zu realisieren, wird der Ausgang und die Anzahl Samples verzögert, die in der Zukunft liegen würden.\\
	 Geht die Impulsantwort bis ins unendliche in die Zukunft, so wird diese einfach abgeschnitten, wenn die Koeffizienten genug klein geworden sind $\rightarrow$ Approximation
	\end{itemize}

	

	


